\section{Appendix}

This section defines many terms that were previously mentioned but not sufficiently
explained in alphabetical order.

\begin{itemize}
    \item \textbf{Constant Folding} is an optimization technique employed by the
    compiler to evaluate constant expressions at compile-time rather than runtime.
    For example, arithmetic expressions such as \texttt{int area = 40 * 50;} which
    will substitute the RHS with the result of this computation, i.e., \texttt{2000}.
    \item \textbf{Function Inlining} is a compiler optimization that replaces a
    function call site with the implementation of the function. In contrast, a
    macro substitution takes place one step earlier, during the preprocessor stage
    which effectively changes the source code processed by the compiler in memory.
    \item \textbf{IPA} (Interprocedural Analysis) is a mechanism that performs
    optimizations across the whole program which includes, among other things: CS
    (Code Straightening), PP (Program Partitioning), IPAA (Interprocedural Pointer
    Alias Analyses), ICP (Interprocedural Copy Propagation). Generally speaking,
    it refers to gathering information about the entire program instead of a single
    procedure.
    \item \textbf{IPO} (Interprocedural Optimization) is an automatic multi-step
    process that employs the compiler to analyze and assess which optimizations
    the code could benefit from. The compiler can choose from a variety of
    optimization techniques\footnote{This is a non-exhaustive list of optimizations
    that is available to the compiler.}: ADP (Array Dimension Padding), CP (Common
    Propagation), PDCE (Partial Dead Call Elimination), WPA (Whole Program Analysis).
    As opposed to IPA, it concerns itself with program transformations that involve
    more than one procedure in a program.
    \item \textbf{LTO} (Link Time Optimization) is a method for achieving better
    runtime performance through whole-program analysis and cross-module optimization.
    During the compile phase, Clang will emit LLVM bitcode files and invokes LLVM
    during the link to generate the final objects that will constitute the executable.
    The LLVM implementation loads all input bitcode files and merges them together
    to produce a single Module. The interprocedural analysis (IPA) as well as the
    interprocedural optimizations (IPO) are performed serially on this monolithic
    module \autocite{johnson2016}.
    \item \textbf{NRVO} (Named Return Value Optimization) is a compiler optimization
    that can remove instantiated intermediary objects that are unique. The compiler
    is able to apply this technique if it is able to determine an object's memory
    location inside a function where all paths return a unique object.
    \item \textbf{Peephole Optimization} is a compiler optimization technique that
    takes a small set of compiler-generated instructions (dicitur peephole) and
    replaces it with a functionally equivalent, but more performant set of instructions.
    \item \textbf{RVO} (Return Value Optimization) is a compiler optimization that
    involves eliminating the temporary object created to hold a function's return
    value.
\end{itemize}