\section{Oroonoko}

\begin{displayquote}
Written by spy, traveler and groundbreaking woman writer Aphra Behn, this story 
of an African prince sold into slavery is considered one of the earliest English 
novels. \autocite{behn2003}
\end{displayquote}

While long prose fictional narratives are not an invention of the eighteenth 
century, Aphra Behn’s \emph{Oroonoko} (which was first published in 1688) is a 
great example for this transitional period where the novel genre was about to 
come into its own. Although we don’t know much about her early life, in part 
because she deliberately obscured details of her upbringing, it is safe to say 
that she is a founding figure for women’s writing who had left behind an 
extensive literary legacy. After the discovery of the new world in 1492 it was 
only a matter of time before the trans-Atlantic slave trade would become one of 
the most profitable businesses in the world. With the decline of the native 
American population, the demand for manpower in the new colonies became 
increasingly strong. This chain of events gave rise to the trans-Atlantic slave 
trade and had a lasting impact on the literary landscape of Europe. It is not by 
accident that many authors of that time turned their attention to the world 
beyond. During this period, she allegedly composed \emph{Oroonoko} “in a few 
hours ... for I never rested my pen a moment for thought.” 
\autocite[2178]{greenblatt2006}. Either by chance or design, Aphra Behn’s novel 
sheds light on the double standards many cultivated, civilized Europeans 
employed, and their hideous character becomes apparent the moment they appear on 
stage. Oroonoko in particular voices his disdain for the foreign faith upon 
realizing that he had been betrayed by the captain he thought a friend:

\begin{displayquote}
Farewell, Sir! It is worth my suffering to gain so true a knowledge both of you 
and of your gods by whom you swear. [...] Come, my fellow-slaves, let us descend 
and see if we can meet with more honour and honesty in the next world we shall 
touch upon. \autocite[58]{behn2003}
\end{displayquote}

Even though Oroonoko is born to a Coramantien king, Behn gives an unusual 
account of his character. Reminiscent of a refined western education that rivals 
that of European princes, he is purposefully made look like more European than 
African in order to appeal to a western readership. By the time he fell into 
captivity it was only thanks to his noble appearance that entitled him to a 
better treatment than the other slaves. 

\begin{displayquote}
He was pretty tall, but of a shape the most exact that can be fancied; the most 
famous statuary could not form the figure of a man more admirably turned from 
head to foot. His face was not of that brown, rusty black which most of that 
nation are, but a perfect ebony or polished jet. His eyes were the most awful 
that could be seen, and very piercing; the white of them being like snow, as 
were his teeth. His nose was rising and Roman instead of African and flat. 
His mouth, the finest shaped that could be seen, far from those great turned 
lips which are so natural to the rest of the 
Negroes. \autocite[18]{behn2003}
\end{displayquote}

As the story progresses, it becomes evident that he is at odds with his status 
as a privileged slave. He still associates the elements of European civilization 
with deceit and dishonesty which in the end led to his capture, but he doesn’t 
seem to condemn slavery in and of itself. It was only when his wife became 
pregnant that it struck him that he could no longer bear the emotional toll of 
being enslaved for the sake of his unborn child. Adding to the fact that 
Oroonoko did not suffer nearly as much as his fellow-slaves under his master 
Tefry and was never put to work, it is no small irony that he would later go on 
to deceive the other slaves himself for his own selfish reasons in a passionate 
speech about honor and freedom:

\begin{displayquote}
\emph{And why}, said he, \emph{my dear friends and fellow-sufferers, should we 
be slaves to an unknown people? Have they vanquished us nobly in fight? Have 
they won us in honourable battle? And are we by the chance of war become their 
slaves? This would not anger a noble heart, this would not animate a soldier’s 
soul. No, but we are bought and sold like apes or monkeys, to be the sport of 
women, fools and cowards, and the support of rogues, runagades that have 
abandoned their own countries for raping, murders, theft and vallainies. Do you 
not hear every day how they upbraid each other with infamy of life, below the 
wildest savages? And shall we render obedience to such a degenerate race, who 
have no one human virtue left to distinguish them from the wildest creatures? 
Will you, I say, suffer the lash from such hands?} They all replied with one 
accord, \emph{No, no, no; Caesar has spoke like a great captain, like a great 
king.} \autocite[89-90]{behn2003}
\end{displayquote}

Perhaps it was because of his loss of his royal identity that Oroonoko was bound 
to revolt sooner or later. Considering his noble heritage and reputation, his 
sudden change of mind does not come as a surprise; yet Oroonoko was first and 
foremost only concerned about the fate of his own family. His last stance of 
passive valor is a clever nod to the fact that aspiring generals of his tribe 
proved their worth by contemptuously cutting off parts of their bodies until one 
or the other resigned or died \autocite[86]{behn2003}.

\begin{displayquote}
He had learned to take tobacco, and when he was assured he should die, he 
desired they would give him a pipe in his mouth, ready lighted which they did, 
and the executioner came and first cut off his members and threw them into the 
fire. After that, with an ill-favoured knife, they cut his ears and his nose, 
and burned them; he still smoked on, as if nothing had touched him. Then they 
hacked off one of his arms, and still he bore up, and held his pipe. But at the 
cutting off the other arm, his head sunk, and his pipe dropped, and he gave up 
the ghost without a groan or a reproach. \autocite[111]{behn2003}    
\end{displayquote}

Many literary historians contest the proclaimed biographical nature of 
\emph{Oroonoko} stressed in Behn’s Epistle Dedicatory, but the circumstantial 
details of the story suggest that she combined three older forms of literary 
narration techniques. While she presents \emph{Oroonoko} as a personal account 
of what she had heard and seen, the book also contains a travel narrative in 
three parts encapsulated in a biography \autocite[2179]{greenblatt2006}. As 
opposed to Daniel Defoe who, a few decades later, would establish a range of 
concerns central to the domestic themes of the English novel, Aphra Behn’s 
\emph{Oroonoko} belongs in the tradition of Imperial Romance 
\autocite[139]{peckCoyle2002}. Hence, depending on how the novel genre is 
defined, some people might argue that she shouldn’t be regarded as the first 
English novelist. Nevertheless, the notion of scientific rigor is a dangerous 
path to tread in literature theory.
