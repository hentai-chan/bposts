\section{Communication}

Communication within the team is one of the determining factors of a project's
success. While many people have a solid idea of what good communication constitutes,
there are a few points that are worth highlighting here.

Establish at least two lines of communication with all members of the team, the
reason being that your primary communication channel is always running at risk
of going going abruptly offline due to server outage or technical malfunctions.
This is especially crucial in emergency situations. Unlike many other lines of
works, the work of an engineer\footnote{Going forward, engineer refers to anyone
working in our value stream, not just developers.} is foremost of mental nature.
Many studies suggest that being constantly exposed to notifications from colleagues
and clients harms the productivity of the team \autocite{harvard2009}. Excessive
communication makes it easier to get carry away by extraneous affairs. It is therefore
recommended to allocate predictable time-offs to each member to designate periods
of time in which they are allowed to remain unresponsive to messages and calls.
In the same vein, open office spaces should be avoided because they are equally
damaging to the productivity when it comes to work that requires long hours of
uninterruptible concentration \autocite{yermolaieva2020} \autocite{deepwork2016}.

In contrast to this train of thought, it is necessary to identify people in your
team that have trouble to communicate problems early on. The best-case scenario
prescribes that information on progress and failure should be communicated in
time. According to a survey conducted in 2016 \autocite{yermolaieva2020}, communication
challenges in geographically distributed Agile development teams appear to be caused
largely by time-zone differences and the team configuration. Social exchange theory
suggests that predefined goals can help the team understand the value of knowing
communication by virtue of knowing that every person involved has a self-interest
in staying engaged and obtaining meaningful results. This also implies that management
needs to cut down the total amount of meetings that take place between work if
there exists a more efficient alternative. While written communication provides
the best way of keeping a record, the absence of direct communication can lead to
low-cohesion relationships, as a result of which people are more prone to dismiss
the opinions and experience of their colleagues. As a project manager, it is your
responsibility to address these sort of issues as soon as they rise to the surface.
