\section{Directory and File Names}

Even though this may seem like a trivial matter at first, there are a few rules
one needs to follow to not run into any cross-platform compatibility issues. In
fact, not complying to these rules willingly may sometimes cause security issues
as any shell metacharacter\footnote{Characters that must be escaped in a shell
script before they can be used as an ordinary character are termed shell metacharacters.}
in a filename will be executed (unescaped) by another shell script \autocite{wheeler2020}
\autocite{marti2014}. See also \url{https://en.wikipedia.org/wiki/Filename} for
a comprehensive list of filename limitations put in place by various operating
systems. In general, prefer UTF-8 encoded lowercase filenames across all file systems.
Using white space characters in directory and filenames is also better to be avoided
because many Unix/Linux shell scripts presume that there are no white space characters
in filenames. For instance, in June of 2021, Microsoft's Ubuntu repositories broke
due to white spaces filenames \autocite{jarina2021}. Moreover, the POSIX standard
may reserve a subset of all possible filenames that consists of identifiers that
begin with an underscore and continue with either another underscore or a capital
letter in the global namespace. The class of names that contains all names matching
this restriction can be expressed by the following regular expression \autocite{oracle2010}:
\path{_[_A-Z][0-9_a-zA-Z]*}. For similar reasons, you should never use a hyphen
as the first character in a filename because by convention, programs accepts their
options as their first argument, usually preceded by a dash \autocite{unixhater1994}.
On top of that, Windows also reserves a list of filenames for internal use only:

\begin{itemize}
    \item \texttt{CON}
    \item \texttt{AUX}
    \item \texttt{NUL}
    \item \texttt{COM1}, \texttt{COM2}, \texttt{COM3}, \texttt{COM4}, \texttt{COM5},
          \texttt{COM6}, \texttt{COM7}, \texttt{COM8}, \texttt{COM9}
    \item \texttt{LPT1}, \texttt{LPT2}, \texttt{LPT3}, \texttt{LPT4}, \texttt{LPT5},
          \texttt{LPT6}, \texttt{LPT7}, \texttt{LPT8}, \texttt{LPT9}
\end{itemize}

This limitation also applies to lowercase deviations of these filenames followed
by an extension \autocite{microsoft2020a}. Finally, using short and concise filenames
is also strongly encouraged to comply with Windows' default maximum path length
of 260 characters (as opposed to Linux' maximum path length of 4096 characters on
most prominent distributions). While it is possible to opt into long paths to lift
this limitation on Windows starting with Windows 10 (Version 1607 and later
\autocite{microsoft2020b}), it is not reasonable to assume that end-users have
enabled this feature on their machine.
