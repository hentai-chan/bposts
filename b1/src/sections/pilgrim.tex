\section{The Pilgrim’s Progress from This World, to That Which Is to Come}

\begin{displayquote}
The Pilgrim’s Progress from This World, to That Which Is to Come is a 1678 
Christian allegory written by John Bunyan. It is regarded as one of the most 
significant works of religious English literature, has been translated into more 
than 200 languages, and has never been out of print. It has also been cited as 
the first novel written in English. Bunyan began his work while in the 
Bedfordshire county prison for violations of the Conventicle Act, which 
prohibited the holding of religious services outside the auspices of the 
established church of England. Early Bunyan scholars such as John Brown believed 
The Pilgrim’s Progress was begun in Bunyan’s second, shorter imprisonment for 
six months in 1675, but more recent scholars such as Roger Sharrock believe that 
it was begun during Bunyan’s initial, more lengthy imprisonment from 1660 to 
1672 right after he had written his spiritual autobiography, Grace Abounding to 
the Chief of Sinners. The English text comprises $108,260$ words and is divided 
into two parts, each reading as a continuous narrative with no chapter 
divisions. The first part was completed in 1677 and entered into the Stationer’s 
Register on 22 December 1677. It was licensed and entered in the “Term 
Catalogue” on 18 February 1678, which is looked upon as the date of first 
publication. After the first edition of the first part in 1678, an expanded 
edition, with additions written after Bunyan was freed, appeared in 1679. The 
second part appeared in 1684. There were eleven editions of the first part in 
John Bunyan’s lifetime, published in successive years from 1678 to 1685 and in 
1688, and there were two editions of the second part, published in 1684 and 
1686. \autocite{bunyan2018}    
\end{displayquote}

Regarded as one of the most famous of world classics and translated into more 
than a hundred languages and dialects, \emph{The Pilgrim’s Progress} claim to 
fame is grounded in its intensely sincere presentation of a Christian’s journey 
to the Celestial City. Almost half a century after the release of the 
\emph{King James Bible}, John Bunyan’s distinctive style draws heavily from the 
language of the reissued bible that is widely believed to be a masterpiece 
of English literature \autocite[70]{fletcher2007}. The story of 
\emph{The Pilgrim’s Progress} therefore easily gives away the author’s 
preference for the protagonist, which takes a toll on the suspense of the 
narration. Predominated by his Puritan conscience, Bunyan repeatedly emphasizes 
that this story is a dream brought into being by his triumphant and loving joy 
in his religion. That also explains the absence of genuine individuals and 
secondary actions in his book, which are completely replaced by type characters 
and a linear narrative that is perfectly intelligible to any child. While it is 
the case that Bunyan’s allegory puts more emphasis on his abundance of faith and 
love than avoiding technical faults in his narration, it was perhaps only for 
that reason that enabled him to attain a poetic beauty and eloquence that is on 
par with trained literary artists \autocite[80]{fletcher2007}. Even for people 
of other faiths there is a lesson to be learnt from reading his works, for all 
men experience struggle throughout their life. There is a phrase that captures 
the importance of friendship quite well: “You are the company you keep”, and if 
it had not been for Hope’s assurance, Christian’s journey could have taken a 
turn for the worse on more than one occasion:

\begin{displayquote}
\{289\} HOPE. My brother, said he, rememberest thou not how valiant thou hast 
been heretofore? Apollyon could not crush thee, nor could all that thou didst 
hear, or see, or feel, in the Valley of the Shadow of Death. What hardship, 
terror, and amazement hast thou already gone through! 
\autocite[64]{bunyan2018}    
\end{displayquote}

For all intents and purposes, \emph{The Pilgrim’s Progress} is a story of 
maintaining hope in view of seemingly insurmountable difficulties that life 
throws at you, but also a zealous attempt to spread the word of God that goes 
back to the roots of Christianity:

\begin{displayquote}
CHR. Yes, that is a good heart that hath good thoughts, and that is a good life 
that is according to God’s commandments; but it is one thing, indeed, to have 
these, and another thing only to think so. \autocite[81]{bunyan2018}    
\end{displayquote}

In contemplation of world treasures and pleasures, Ignorance fails to understand 
that accumulating wealth goes against the teachings of his very own belief and 
is thus further foreshadowing his fate by refusing to accept the good counsel 
presented to him by Christ and Hope. This is also becoming more prominent in 
modern times where constant exposure to other people’s best snapshot in life 
through social media gives an account of what we as a society seem to value the 
most. Then again, the philosophy of human nature is much more complex than the 
type characters portrayed in \emph{The Pilgrim’s Progress}, but this does 
nothing to alter the fact that Bunyan’s work was ahead of his time. Almost two 
centuries later, Louisa May Alcott’s \emph{Little Women} bears testimony to 
the influence John Bunyan held over the fellowship of the cross for years to 
come.

\begin{flushleft}
    \begin{displayquote}
    Go then, my little Book, and show to all                \linebreak
    That entertain, and bid thee welcome shall,             \linebreak
    What thou dost keep close shut up in thy breast;        \linebreak
    And wish what thou dost show them choose to be blest    \linebreak
    To them for good, may make them choose to be            \linebreak
    Pilgrim’s better, by far, than thee or me,              \linebreak
    Tell them of Mercy; she is one                          \linebreak
    Who early hath her pilgrimage begun.                    \linebreak
    Yea, let young damsels learn of her to prize            \linebreak
    The world which is to come, and so be wise;             \linebreak
    For little tripping maids may follow God                \linebreak
    Along the ways which saintly feet have trod.            \linebreak
    \autocite[Preface]{alcott2014}                           
    \end{displayquote}
\end{flushleft}

This book unfolds the story of four young girls growing up in a small community 
behind the events of the American Civil War. Brought up in poverty, the March 
girls learned that complaining about one’s dissatisfaction can be both, a force 
that promotes and hinders personal growth. In many ways, each and everyone of us 
carries a burden in some form or another, and to overcome our little 
imperfections sometimes means to fail over and over again before we finally 
succeed. But this is just part of the journey and highlights the importance of 
personal experience from which we draw the courage to carry on. While 
\emph{Little Women} is a book that is primarily meant to guide and entertain 
young children, there is something beautiful to discover for adults in a world 
where materialism has eclipsed religion. After all, life is best enjoyed in 
moderation even if it feels at times impossible to find content in everyday 
life. 

\begin{displayquote}
Mrs. March broke the silence that followed Jo’s words, by saying in her cherry 
voice, “Do you remember how you used to play Pilgrim’s Progress when you were 
little things? Nothing delighted you more than have me tie you hats and sticks, 
and rolls of paper, and let you travel through the house from the cellar, which 
was the City of Destruction, up, up, to the house-top, where you had all the 
lovely things you could collect to make a Celestial City.”

“What fun it was, especially going by the lions, fighting Apollyon, and passing 
through the Valley where the hobgoblins were!” said Jo.

“I liked the place where the bundles fell off and tumbled downstairs,” said Meg.

“My favorite part was when we came out on the flat roof where our flowers and 
arbors, and pretty things were, and all stood and sung for joy up there in the 
sunshine,” said Beth, smiling, as if that pleasant moment had come back to her.

“I don’t remember much about it, except that I was afraid of the cellar and the 
dark entry, and always liked the cake and milk we had up at the top. If I wasn’t 
too old for such things, I’d rather like to play it over again,” said Amy, who 
began to talk of renouncing childish things at the mature age of twelve. 
\autocite[14]{alcott2014}
\end{displayquote}

Looking back, we grew up in a short passage of time, but being an adult doesn’t 
turn you into a know-it-all overnight. On the same note, becoming more mature 
doesn’t mean you have to cast away your imagination. By taking turns at being 
children it becomes easier to pass on experiences and create an environment 
where we feel secure to learn from each other. Although many famous authors are 
dead by now, their stories live on for anyone who reads beyond the first page.
