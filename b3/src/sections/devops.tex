\section{DevOps Principles}

In DevOps\footnote{The term DevOps was first coined by Patrick Debois in 2009 during
the first DevOpsDays in Ghent, Belgium.}, the term value stream is defined as the
process required to convert a business hypothesis into a technology-enabled service
that delivers value to the customer. One key point to take away from the three ways
is to turn our attention to reducing the overall lead time throughout the development
cycle. After all, the proportion of process time to lead time serves as an important
measure of efficiency. To enable fast flow and high quality, small batches of work
going through the design/development and tests/operations value stream should be
favored over big and potentially disruptive changes because value generation only
starts once code passes the UAT (User Acceptance Test) in an production environment.
A core tenet of virtually all modern process improvements is to continually shorten
and amplify our feedback loops. As complex tasks suffer greatly from multitasking,
incurring all costs of re-establishing context as well as cognitive rules and goals,
it is recommended to implement a limit by codifying and enforcing an upper bound
for WIP for each work center \autocite{devops2016}.

\begin{enumerate}
    \item Ready
    \item Investigate
    \item In Progress
    \item Done
    \item UAT
    \item Delivered
\end{enumerate}

% NOTE: an illustration would work great wonders here!
Reducing the batch size of each ticket that goes through this cycle results in
faster lead times nd error detection, but also less WIP and rework.
