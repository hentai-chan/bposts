\section{Conclusions}

While it is not wrong to say that Python is an interpreted language, this is not
the whole story. The truth is that it is both an interpreted and compiled language.
Because the PVM interpretates the bytecode instructions line-by-line, it is able
to immediately react to changes made in the source code, though this interpretation
step prolongs the program execution.

As opposed to GCC or Clang, the Python interpreter executes bytecode rather than
machine code instructions. In this case, tradeoffs are made between the time it
takes to analyze the source code and the overall execution time. Since no intermediate
object code is generated, Python code is generally slower than compared to statically
typed languages like C++ because it requires type conversion validation at runtime
\autocite{zehra2020}. Moreover, since Python facilitates the use of a GC, it takes
longer for the program to identify and allocate free memory. On top of that, the
GC is also not thread-safe which is the primary reason why Python requires a GIL.
Dynamically typed languages in general take longer to execute because they cannot
employ the benefits of platform-specific machine instructions that were optimized
for a particular architecture. 
