\section{Robinson Crusoe}

\begin{displayquote}
Regarded as the first English novel, Robinson Crusoe is a work that goes to the 
heart of human existence. Told through the journal of Crusoe, the sole survivor 
of a shipwreck, it chronicles his daily battle to stay alive on a dessert 
island, where his greatest struggle is with solitude – until a single footprint 
appears in the sand. Vividly depicting an individual’s psychological development 
from terrified survivor to master of man and nature, Defoe created one of the 
most enduring, universal myths in literature. \autocite{defoe2012}
\end{displayquote}

In Daniel Defoe’s \emph{Robinson Crusoe}, which first appeared in 1719, we find 
a very popular contender for the first English novel. As far as the Western 
Hemisphere is concerned, the early eighteenth century was a turning point for 
middle-class people who, up until then, had their daily lives influenced and 
organized by the teachings of Christianity. Amidst a rapidly advancing 
commercial and cultural society that emerged in the British townscape, these 
provincial centers began to flourish with new theaters, assembly rooms, 
libraries, Freemason lodges and coffeehouses \autocite{britannica2020}. At its 
heart, \emph{Robinson Crusoe} carefully mirrors the zeitgeist of the urban 
population and reveals the underlying anxiety that affected many people at that 
time, as their religious dimension in life continued to slowly turn into 
indifference. Embedded in a beautifully observed environment, Defoe was able to 
skillfully question the new concept of the self that took shape in the 
consciousness of the working class. This restless entrepreneurial spirit that 
came along with the expansion of the first British Empire and the establishment 
of the joint-stock companies, most notably the East India company, is casted 
back into Robinson Crusoe’s character and culminates in a shipwreck on a dessert 
island, where he eventually assumes the role of a governor. During all this time 
Defoe not only draws attention to the balance of the relationship between human 
beings and God, but also to the relationship of racial and colonial superiority. 
An illustration of this imbalance can be found in the relationship between 
Crusoe and Friday, on whom he first imposes his language and religious belief, 
but later choose to replace with mutual respect and friendship. Even though the 
narrative structure of the book is simple in nature, Defoe was able to weave in 
complex-layered elements of English society and writes into existence a new 
class of people in the language of actual speech \autocite[187]{peckCoyle2002}. 
Without a doubt is Daniel Defoe one of the most influential English writers who 
shaped our image of the eighteenth century and its society 
\autocite[2289]{greenblatt2006}.
