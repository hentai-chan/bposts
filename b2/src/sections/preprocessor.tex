\section{Preprocessor}

A preprocessor directive uses the pound character (\path{#}) to designate a
preprocessor statement. These directives give rise to various events, such as

\begin{itemize}
    \item file inclusion (\path{#include}) in which the file being processed
    incorporates the content of another file.
    \item macro substitution (\path{#define}) to denote a sequence of text that
    is to be replaced by a definition.
    \item preprocessor directives (\path{#ifdef} or\footnote{These two are not
    exactly the same, see also \url{https://stackoverflow.com/questions/135069}.}
    \path{#if defined}) for introducing conditions under which segments of code
    shan't compile\footnote{This directive is primarily used for header guards,
    but it can also be used to denote sections in code that are specific to the
    operating system, for example \path{#if defined(__ANDROID__)}.}. 
\end{itemize}

The preprocessor takes precedence over the compiler. Once this step is executed,
all preprocessor directives are removed from the source file. Files that go through
this translation process turn into translation units that only exist in memory.