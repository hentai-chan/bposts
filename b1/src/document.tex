% !TEX spellcheck=en_US
\documentclass[a4paper]{article}
\usepackage[american]{babel}
\usepackage[T1]{fontenc}
\usepackage[utf8]{inputenc}
\usepackage[activate={true,nocompatibility},final,tracking=true,kerning=true,spacing=true,factor=1100,stretch=10,shrink=10]{microtype}
\usepackage{graphicx}
\usepackage{csquotes}
\usepackage{hyperref}
\usepackage[backend=biber,style=mla]{biblatex}

% misc macros
\newcommand*{\pref}[2]{#1 (\ref{#2})}

% microtype
\SetTracking{encoding={*}, shape=sc}{0}

\addbibresource{bibliography.bib}

\begin{document}

\begin{titlepage}
    \centering
    {\scshape\Huge \bfseries{University of Applied Sciences} \par}
    \par\vspace{1cm}
    \includegraphics[width=0.3\textwidth]{images/bwhtwlogo.jpg}\par
    \vspace{3cm}
    {\huge\bfseries The Early English Novels \par}
    \vspace{1cm}
    {\LARGE\itshape Shinonome Mazawa \par}
    \vspace{1cm}
    {\large\today\par}
    \vfill
    \begin{abstract}
        Through the example of three popular contenders for the first English novel, 
        this document serves as an introduction to the novel genre and how it first 
        came into being.  
    \end{abstract}
\end{titlepage}

\newpage

\microtypesetup{protrusion=false}
\tableofcontents
\microtypesetup{protrusion=true}

\newpage

% === Begin Section Includes ===

\section{Introduction}

For many developers who have primarily worked with interpreted languages such as
Python or JavaScript in the past, learning C++ may seem like a daunting task at
first: There're many moving parts between your \path{hello_world.cpp} and final
executable that you can't simply ignore, especially if you're keen on developing
cross-platform programs or want to collaborate with developers who use a different
operating system. In this document you will learn what is happening behind the scenes
when you build a C++ project. After reading this text you should be able to explain
the build process more precisely without going into implementation details. It also
serves as a starting point in case you want to learn more about compilers. 


\section{Terminology}

Poetry is a literary genre characterized by rhythmical patterns of a language 
and usually employs meters, syllabifications, rhymes, alliterations or any 
combination of these elements. In contrast, any material that is not written 
in rhythmical patterns is considered prose. Many modern genres fall under this 
form, such as short stories, novellas or novels. Thirdly and for the sake of 
completeness, drama as we know it today is a play that often consists of three 
acts and a little disregard for Aristotelian rules involving verisimilitude. 
This is quite different from traditional Greek drama, as defined by Aristotle, 
in which a play consists of five acts and adheres to the three dramatic unities: 
unity of action, unity of time, and unity of place. But here we will concern 
ourselves primarily with novels and its various subgenres. In the broadest sense
of the term, any extended fictional prose narrative focusing on a few primary 
characters that by times involves a score of secondary characters is generally 
thought of as a novel in the realms of English literature. In addition to this
rough definition, some people like to set an arbitrarily count of $50,000$ words 
or more in order to draw a line between short stories, novellas and novels 
\autocite{wheeler2018}.

\begin{table}[h]
    \centering
    \begin{tabular}{|c|c|c|c|}
        \hline
        Genre & Short Story & Novella & Novel \\
        \hline
        Word Count & $<7,500$ & $20,000<$ & $50,000<$\\
        \hline
    \end{tabular}
    \caption{Commonly used word count for works of fiction}\label{wordcount}
\end{table}

Keep in mind that these word counts are not set in stone and their only purpose 
is to create order in the chaos of vague definitions that are part of the very 
nature of literature theory. There are even more works of fiction such as flash 
stories and novelettes, but that is outside the scope of this discussion. Last 
but not least I want to point out that the word count is not the only defining 
characteristic, as will be shown later. In my private book collection, I have 
found at least three books that claim to be among the first English novels ever 
written. But different notions of the term novel further complicate the search 
for a definite answer. The following excerpts are taken directly from the 
back-cover text of my books, which are accompanied by a brief description of 
the author’s accomplishments.


\section{Oroonoko}

\begin{displayquote}
Written by spy, traveler and groundbreaking woman writer Aphra Behn, this story 
of an African prince sold into slavery is considered one of the earliest English 
novels. \autocite{behn2003}
\end{displayquote}

While long prose fictional narratives are not an invention of the eighteenth 
century, Aphra Behn’s \emph{Oroonoko} (which was first published in 1688) is a 
great example for this transitional period where the novel genre was about to 
come into its own. Although we don’t know much about her early life, in part 
because she deliberately obscured details of her upbringing, it is safe to say 
that she is a founding figure for women’s writing who had left behind an 
extensive literary legacy. After the discovery of the new world in 1492 it was 
only a matter of time before the trans-Atlantic slave trade would become one of 
the most profitable businesses in the world. With the decline of the native 
American population, the demand for manpower in the new colonies became 
increasingly strong. This chain of events gave rise to the trans-Atlantic slave 
trade and had a lasting impact on the literary landscape of Europe. It is not by 
accident that many authors of that time turned their attention to the world 
beyond. During this period, she allegedly composed \emph{Oroonoko} “in a few 
hours ... for I never rested my pen a moment for thought.” 
\autocite[2178]{greenblatt2006}. Either by chance or design, Aphra Behn’s novel 
sheds light on the double standards many cultivated, civilized Europeans 
employed, and their hideous character becomes apparent the moment they appear on 
stage. Oroonoko in particular voices his disdain for the foreign faith upon 
realizing that he had been betrayed by the captain he thought a friend:

\begin{displayquote}
Farewell, Sir! It is worth my suffering to gain so true a knowledge both of you 
and of your gods by whom you swear. [...] Come, my fellow-slaves, let us descend 
and see if we can meet with more honour and honesty in the next world we shall 
touch upon. \autocite[58]{behn2003}
\end{displayquote}

Even though Oroonoko is born to a Coramantien king, Behn gives an unusual 
account of his character. Reminiscent of a refined western education that rivals 
that of European princes, he is purposefully made look like more European than 
African in order to appeal to a western readership. By the time he fell into 
captivity it was only thanks to his noble appearance that entitled him to a 
better treatment than the other slaves. 

\begin{displayquote}
He was pretty tall, but of a shape the most exact that can be fancied; the most 
famous statuary could not form the figure of a man more admirably turned from 
head to foot. His face was not of that brown, rusty black which most of that 
nation are, but a perfect ebony or polished jet. His eyes were the most awful 
that could be seen, and very piercing; the white of them being like snow, as 
were his teeth. His nose was rising and Roman instead of African and flat. 
His mouth, the finest shaped that could be seen, far from those great turned 
lips which are so natural to the rest of the 
Negroes. \autocite[18]{behn2003}
\end{displayquote}

As the story progresses, it becomes evident that he is at odds with his status 
as a privileged slave. He still associates the elements of European civilization 
with deceit and dishonesty which in the end led to his capture, but he doesn’t 
seem to condemn slavery in and of itself. It was only when his wife became 
pregnant that it struck him that he could no longer bear the emotional toll of 
being enslaved for the sake of his unborn child. Adding to the fact that 
Oroonoko did not suffer nearly as much as his fellow-slaves under his master 
Tefry and was never put to work, it is no small irony that he would later go on 
to deceive the other slaves himself for his own selfish reasons in a passionate 
speech about honor and freedom:

\begin{displayquote}
\emph{And why}, said he, \emph{my dear friends and fellow-sufferers, should we 
be slaves to an unknown people? Have they vanquished us nobly in fight? Have 
they won us in honourable battle? And are we by the chance of war become their 
slaves? This would not anger a noble heart, this would not animate a soldier’s 
soul. No, but we are bought and sold like apes or monkeys, to be the sport of 
women, fools and cowards, and the support of rogues, runagades that have 
abandoned their own countries for raping, murders, theft and vallainies. Do you 
not hear every day how they upbraid each other with infamy of life, below the 
wildest savages? And shall we render obedience to such a degenerate race, who 
have no one human virtue left to distinguish them from the wildest creatures? 
Will you, I say, suffer the lash from such hands?} They all replied with one 
accord, \emph{No, no, no; Caesar has spoke like a great captain, like a great 
king.} \autocite[89-90]{behn2003}
\end{displayquote}

Perhaps it was because of his loss of his royal identity that Oroonoko was bound 
to revolt sooner or later. Considering his noble heritage and reputation, his 
sudden change of mind does not come as a surprise; yet Oroonoko was first and 
foremost only concerned about the fate of his own family. His last stance of 
passive valor is a clever nod to the fact that aspiring generals of his tribe 
proved their worth by contemptuously cutting off parts of their bodies until one 
or the other resigned or died \autocite[86]{behn2003}.

\begin{displayquote}
He had learned to take tobacco, and when he was assured he should die, he 
desired they would give him a pipe in his mouth, ready lighted which they did, 
and the executioner came and first cut off his members and threw them into the 
fire. After that, with an ill-favoured knife, they cut his ears and his nose, 
and burned them; he still smoked on, as if nothing had touched him. Then they 
hacked off one of his arms, and still he bore up, and held his pipe. But at the 
cutting off the other arm, his head sunk, and his pipe dropped, and he gave up 
the ghost without a groan or a reproach. \autocite[111]{behn2003}    
\end{displayquote}

Many literary historians contest the proclaimed biographical nature of 
\emph{Oroonoko} stressed in Behn’s Epistle Dedicatory, but the circumstantial 
details of the story suggest that she combined three older forms of literary 
narration techniques. While she presents \emph{Oroonoko} as a personal account 
of what she had heard and seen, the book also contains a travel narrative in 
three parts encapsulated in a biography \autocite[2179]{greenblatt2006}. As 
opposed to Daniel Defoe who, a few decades later, would establish a range of 
concerns central to the domestic themes of the English novel, Aphra Behn’s 
\emph{Oroonoko} belongs in the tradition of Imperial Romance 
\autocite[139]{peckCoyle2002}. Hence, depending on how the novel genre is 
defined, some people might argue that she shouldn’t be regarded as the first 
English novelist. Nevertheless, the notion of scientific rigor is a dangerous 
path to tread in literature theory.


\section{The Pilgrim’s Progress from This World, to That Which Is to Come}

\begin{displayquote}
The Pilgrim’s Progress from This World, to That Which Is to Come is a 1678 
Christian allegory written by John Bunyan. It is regarded as one of the most 
significant works of religious English literature, has been translated into more 
than 200 languages, and has never been out of print. It has also been cited as 
the first novel written in English. Bunyan began his work while in the 
Bedfordshire county prison for violations of the Conventicle Act, which 
prohibited the holding of religious services outside the auspices of the 
established church of England. Early Bunyan scholars such as John Brown believed 
The Pilgrim’s Progress was begun in Bunyan’s second, shorter imprisonment for 
six months in 1675, but more recent scholars such as Roger Sharrock believe that 
it was begun during Bunyan’s initial, more lengthy imprisonment from 1660 to 
1672 right after he had written his spiritual autobiography, Grace Abounding to 
the Chief of Sinners. The English text comprises $108,260$ words and is divided 
into two parts, each reading as a continuous narrative with no chapter 
divisions. The first part was completed in 1677 and entered into the Stationer’s 
Register on 22 December 1677. It was licensed and entered in the “Term 
Catalogue” on 18 February 1678, which is looked upon as the date of first 
publication. After the first edition of the first part in 1678, an expanded 
edition, with additions written after Bunyan was freed, appeared in 1679. The 
second part appeared in 1684. There were eleven editions of the first part in 
John Bunyan’s lifetime, published in successive years from 1678 to 1685 and in 
1688, and there were two editions of the second part, published in 1684 and 
1686. \autocite{bunyan2018}    
\end{displayquote}

Regarded as one of the most famous of world classics and translated into more 
than a hundred languages and dialects, \emph{The Pilgrim’s Progress} claim to 
fame is grounded in its intensely sincere presentation of a Christian’s journey 
to the Celestial City. Almost half a century after the release of the 
\emph{King James Bible}, John Bunyan’s distinctive style draws heavily from the 
language of the reissued bible that is widely believed to be a masterpiece 
of English literature \autocite[70]{fletcher2007}. The story of 
\emph{The Pilgrim’s Progress} therefore easily gives away the author’s 
preference for the protagonist, which takes a toll on the suspense of the 
narration. Predominated by his Puritan conscience, Bunyan repeatedly emphasizes 
that this story is a dream brought into being by his triumphant and loving joy 
in his religion. That also explains the absence of genuine individuals and 
secondary actions in his book, which are completely replaced by type characters 
and a linear narrative that is perfectly intelligible to any child. While it is 
the case that Bunyan’s allegory puts more emphasis on his abundance of faith and 
love than avoiding technical faults in his narration, it was perhaps only for 
that reason that enabled him to attain a poetic beauty and eloquence that is on 
par with trained literary artists \autocite[80]{fletcher2007}. Even for people 
of other faiths there is a lesson to be learnt from reading his works, for all 
men experience struggle throughout their life. There is a phrase that captures 
the importance of friendship quite well: “You are the company you keep”, and if 
it had not been for Hope’s assurance, Christian’s journey could have taken a 
turn for the worse on more than one occasion:

\begin{displayquote}
\{289\} HOPE. My brother, said he, rememberest thou not how valiant thou hast 
been heretofore? Apollyon could not crush thee, nor could all that thou didst 
hear, or see, or feel, in the Valley of the Shadow of Death. What hardship, 
terror, and amazement hast thou already gone through! 
\autocite[64]{bunyan2018}    
\end{displayquote}

For all intents and purposes, \emph{The Pilgrim’s Progress} is a story of 
maintaining hope in view of seemingly insurmountable difficulties that life 
throws at you, but also a zealous attempt to spread the word of God that goes 
back to the roots of Christianity:

\begin{displayquote}
CHR. Yes, that is a good heart that hath good thoughts, and that is a good life 
that is according to God’s commandments; but it is one thing, indeed, to have 
these, and another thing only to think so. \autocite[81]{bunyan2018}    
\end{displayquote}

In contemplation of world treasures and pleasures, Ignorance fails to understand 
that accumulating wealth goes against the teachings of his very own belief and 
is thus further foreshadowing his fate by refusing to accept the good counsel 
presented to him by Christ and Hope. This is also becoming more prominent in 
modern times where constant exposure to other people’s best snapshot in life 
through social media gives an account of what we as a society seem to value the 
most. Then again, the philosophy of human nature is much more complex than the 
type characters portrayed in \emph{The Pilgrim’s Progress}, but this does 
nothing to alter the fact that Bunyan’s work was ahead of his time. Almost two 
centuries later, Louisa May Alcott’s \emph{Little Women} bears testimony to 
the influence John Bunyan held over the fellowship of the cross for years to 
come.

\begin{flushleft}
    \begin{displayquote}
    Go then, my little Book, and show to all                \linebreak
    That entertain, and bid thee welcome shall,             \linebreak
    What thou dost keep close shut up in thy breast;        \linebreak
    And wish what thou dost show them choose to be blest    \linebreak
    To them for good, may make them choose to be            \linebreak
    Pilgrim’s better, by far, than thee or me,              \linebreak
    Tell them of Mercy; she is one                          \linebreak
    Who early hath her pilgrimage begun.                    \linebreak
    Yea, let young damsels learn of her to prize            \linebreak
    The world which is to come, and so be wise;             \linebreak
    For little tripping maids may follow God                \linebreak
    Along the ways which saintly feet have trod.            \linebreak
    \autocite[Preface]{alcott2014}                           
    \end{displayquote}
\end{flushleft}

This book unfolds the story of four young girls growing up in a small community 
behind the events of the American Civil War. Brought up in poverty, the March 
girls learned that complaining about one’s dissatisfaction can be both, a force 
that promotes and hinders personal growth. In many ways, each and everyone of us 
carries a burden in some form or another, and to overcome our little 
imperfections sometimes means to fail over and over again before we finally 
succeed. But this is just part of the journey and highlights the importance of 
personal experience from which we draw the courage to carry on. While 
\emph{Little Women} is a book that is primarily meant to guide and entertain 
young children, there is something beautiful to discover for adults in a world 
where materialism has eclipsed religion. After all, life is best enjoyed in 
moderation even if it feels at times impossible to find content in everyday 
life. 

\begin{displayquote}
Mrs. March broke the silence that followed Jo’s words, by saying in her cherry 
voice, “Do you remember how you used to play Pilgrim’s Progress when you were 
little things? Nothing delighted you more than have me tie you hats and sticks, 
and rolls of paper, and let you travel through the house from the cellar, which 
was the City of Destruction, up, up, to the house-top, where you had all the 
lovely things you could collect to make a Celestial City.”

“What fun it was, especially going by the lions, fighting Apollyon, and passing 
through the Valley where the hobgoblins were!” said Jo.

“I liked the place where the bundles fell off and tumbled downstairs,” said Meg.

“My favorite part was when we came out on the flat roof where our flowers and 
arbors, and pretty things were, and all stood and sung for joy up there in the 
sunshine,” said Beth, smiling, as if that pleasant moment had come back to her.

“I don’t remember much about it, except that I was afraid of the cellar and the 
dark entry, and always liked the cake and milk we had up at the top. If I wasn’t 
too old for such things, I’d rather like to play it over again,” said Amy, who 
began to talk of renouncing childish things at the mature age of twelve. 
\autocite[14]{alcott2014}
\end{displayquote}

Looking back, we grew up in a short passage of time, but being an adult doesn’t 
turn you into a know-it-all overnight. On the same note, becoming more mature 
doesn’t mean you have to cast away your imagination. By taking turns at being 
children it becomes easier to pass on experiences and create an environment 
where we feel secure to learn from each other. Although many famous authors are 
dead by now, their stories live on for anyone who reads beyond the first page.


% === End Section Includes ====

\newpage

\medskip
\printbibliography

\end{document}
