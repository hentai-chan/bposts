\section{Terminology}

Poetry is a literary genre characterized by rhythmical patterns of a language 
and usually employs meters, syllabifications, rhymes, alliterations or any 
combination of these elements. In contrast, any material that is not written 
in rhythmical patterns is considered prose. Many modern genres fall under this 
form, such as short stories, novellas or novels. Thirdly and for the sake of 
completeness, drama as we know it today is a play that often consists of three 
acts and a little disregard for Aristotelian rules involving verisimilitude. 
This is quite different from traditional Greek drama, as defined by Aristotle, 
in which a play consists of five acts and adheres to the three dramatic unities: 
unity of action, unity of time, and unity of place. But here we will concern 
ourselves primarily with novels and its various subgenres. In the broadest sense
of the term, any extended fictional prose narrative focusing on a few primary 
characters that by times involves a score of secondary characters is generally 
thought of as a novel in the realms of English literature. In addition to this
rough definition, some people like to set an arbitrarily count of $50,000$ words 
or more in order to draw a line between short stories, novellas and novels 
\autocite{wheeler2018}.

\begin{table}[h]
    \centering
    \begin{tabular}{|c|c|c|c|}
        \hline
        Genre & Short Story & Novella & Novel \\
        \hline
        Word Count & $<7,500$ & $20,000<$ & $50,000<$\\
        \hline
    \end{tabular}
    \caption{Commonly used word count for works of fiction}\label{wordcount}
\end{table}

Keep in mind that these word counts are not set in stone and their only purpose 
is to create order in the chaos of vague definitions that are part of the very 
nature of literature theory. There are even more works of fiction such as flash 
stories and novelettes, but that is outside the scope of this discussion. Last 
but not least I want to point out that the word count is not the only defining 
characteristic, as will be shown later. In my private book collection, I have 
found at least three books that claim to be among the first English novels ever 
written. But different notions of the term novel further complicate the search 
for a definite answer. The following excerpts are taken directly from the 
back-cover text of my books, which are accompanied by a brief description of 
the author’s accomplishments.
